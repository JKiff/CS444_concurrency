\documentclass[10pt,a4paper,english]{article}
\usepackage[english]{babel}
\usepackage{longtable,hyperref}
\newcommand{\longtableendfoot}{Please continue at the next page}
\usepackage[utf8]{inputenc}
\usepackage{url}
\usepackage{textcomp}
\usepackage[left=0.75in, right=0.75in, bottom=0.75in]{geometry}
\usepackage{graphicx}	% For including graphics
\newcommand\tab[1][1cm]{\hspace*{#1}}	% Controls tab spacing for /tab
\newenvironment{question}[2][Question]	{\begin{trivlist}
\item[\hskip \labelsep {\bfseries #1}\hskip \labelsep {\bfseries #2.}]}
{\end{trivlist}} % Creates /question object


\begin{document}
%{\fontfamily{lmss}\selectfont

\begin{titlepage}
\vspace*{\fill}

\newcommand{\HRule}{\rule{\linewidth}{0.5mm}} % Defines a new command for the horizontal lines, change thickness here

\center % Center everything on the page

%----------------------------------------------------------------------------------------
%TITLE SECTION
%----------------------------------------------------------------------------------------

{ \huge \bfseries CS444 Assignment 1}\\[0.4cm] % Title of your document

%----------------------------------------------------------------------------------------
%HEADING SECTIONS
%----------------------------------------------------------------------------------------

\textsc{\LARGE Jason Kiff, Terri Hewitt, Jesse Chick, Stephanie Hughes, and Adam Barton}\\[0.5cm] % Name of your university/college
\textsc{\Large Operating Systems II}\\[0.5cm] % Major heading such as course name
\textsc{\large Fall 2018}\\[2.5cm] % Minor heading such as course title


%\HRule \\[0.8cm]
\begin{minipage}{.6\textwidth}
%\HRule \\[0.8cm]
\textbf{\large Abstract} \\
\HRule \\[0.4cm]
Operating Systems II Assignment 1 has our group document our work as we progress. This document will serve as that write-up and provide information on the commands we have executed, answers to questions asked of us, and a log of our commit history from git.
\\[0.4cm]
\HRule \\[1.5cm]
\end{minipage}
%\HRule \\[1.5cm]

                %----------------------------------------------------------------------------------------
                %DATE SECTION
                %-----------------    -----------------------------------------------------------------------

{\large \today}\\[3cm] % Date, change the \today to a set date if you want to be precise

%----------------------------------------------------------------------------------------
%LOGO SECTION
%------   ----------------------------------------------------------------------------------

%\includegraphics{Logo}\\[1cm] % Include a department/university logo - this will require the graphicx package

%----------------------------------------------------------------------------------------

\vfill % Fill the rest of the page with whitespace



\end{titlepage}

\tableofcontents
\newpage

\section{Log of commands used}
\label{sec:introduction}

\subsection{Running Pre-built Kernel}

\begin{itemize}
\item \$ cd $\sim$
\item \$ git clone git://git.yoctoproject.org/linux-yocto-3.19
\item \$ cd linux-yocto-3.19/
\item \$ git checkout tags/v3.19.2
\item \$ cp -r /scratch/files $\sim$
\item \$ cd $\sim$/files
\item\$  source /scratch/opt/environment-setup-i586-poky-linux
\item \$ qemu-system-i386 -gdb tcp::5517 -S -nographic -kernel bzImage-qemux86.bin -drive file=core-image-lsb-sdk-qemux86.ext4,if=virtio -enable-kvm -net none -usb -localtime --no-reboot --append "root=/dev/vda rw console=ttyS0 debug"
\end{itemize}

\subsection{Building the Kernel}

\begin{itemize}
    \item \$ cd /scratch/fall2018
    \item \$ mkdir group17
    \item \$ cd group17/
    \item cp ../../files ./
    \item git clone git://git.yoctoproject.org/linux-yocto-3.19
\item cd linux-yocto-3.19/
\item git checkout tags/v3.19.2
\item cd ../files/
\item source /scratch/opt/environment-setup-i586-poky-linux
\item cd ../linux-yocto-3.19/
\item cp ../files/config-3.19.2-yocto-standard ./.config
\item make -j4 all
\item cd ../files/
\item qemu-system-i386 -gdb tcp::5517 -S -nographic -kernel ../linux-yocto-3.19/arch/x86/boot/bzImage -drive file=core-image-lsb-sdk-qemux86.ext4,if=virtio -enable-kvm -net none -usb -localtime --no-reboot --append "root=/dev/vda rw console=ttyS0 debug"
\end{itemize}

\subsection{Running the Kernel using GDB}

\begin{itemize}
\item cd /scratch/fall2018/group17
\item cd ./files
\item source /scratch/opt/environment-setup-i586-poky-linux
\item qemu-system-i386 -gdb tcp::5517 -S -nographic -kernel bzImage-qemux86.bin -drive file=core-image-lsb-sdk-qemux86.ext4,if=virtio -enable-kvm -net none -usb -localtime --no-reboot --append "root=/dev/vda rw console=ttyS0 debug"
\item Open up a new console window
\item cd /scratch/fall2018/group17
\item cd ./files
\item source /scratch/opt/environment-setup-i586-poky-linux
\item \$GDB
\end{itemize}

\section{Explanation of Flags in qemu Command-Line}
\label{sec:flags}

\subsection{-gdb Flag}
The -gdb flag sets up a gdb server to allow us to connect to the kernel using gdb and the port that is provided as the argument next on the command line.

\subsection{-kernel Flag}
-kernel bzImage use 'bzImage' as kernel image\\
Provide the location of the kernel image to use. The kernel image is a .bz compressed file.

\subsection{-S Flag}
"Freezes" CPU at startup. 'c' can be used to start execution.

\subsection{-nographic Flag}
Disables graphical output and redirects serial I/Os to console.  

\subsection{-drive Flag}
file=file -- use 'file' as a drive image.
The -drive option allows you to specify the location of the drive image you want to use. Additional options are available and supplied via comma-separated-values to specify further information about the drive image. On our command line we have also specified the 'if' option which specifies the type of the file system (virtio).

\subsection{-enable-kvm Flag}
Enables KVM full virtualization support so that operating systems and programs can run on entirely virtual hardware.

\subsection{-net Flag}
Use of '-net' without any options or '-net none' indicates zero network devices.

\subsection{-usb Flag}
Enable the USB driver.

\subsection{-localtime Flag}
Specify the RTC option. Other options are 'utc' and 'date'.

\subsection{--no-reboot Flag}
Exits instead of rebooting. Adding this option will not allow the VM to reboot, so no reboot upon execution.

\subsection{--append Flag}
-append cmdline use 'cmdline' as kernel command line
Adds extra option to the kernel command line.


\section{Concurrency Assignment}
\begin{question}{1}
What do you think the main point of this assignment is?
\end{question}

To refamiliarize ourselves with C. We have had prior experience in OS1 with programming in parallel. This assignment gets us thinking back to our experiences programming in parallel and to demonstrate this through a producer-consumer problem.

\begin{question}{2}
How did you personally approach the problem? Design decisions, algorithm, etc.
\end{question}

We started working on the problem from the bottom up. We started with what we knew how to code, the data structure for the buffer nodes. Then we worked on getting the randomization to work. After we tested that the randomization worked, we started working on creating consumers and producers to add and remove nodes from our buffer. Finally, we added in the assembly code to use rdrand if it can be used on the host. 

\begin{question}{3}
How did you ensure your solution was correct? Testing details, for instance.
\end{question}

To ensure our solution was correct, we tested using various numbers of consumers and producers. We found that using only one consumer and one producer would cause the buffer to remain fairly empty most of the time. We also tested our program using 32 producers and 1 consumer to cause our buffer to reach the maximum limit to ensure that the buffer didn't overflow.

\begin{question}{4}
What did you learn?
\end{question}

We learned more about some of the other random number generators, like rdrand and Mersenne Twister, and the difference between how these functions generate random numbers. In addition, we also discovered how to write in-line assembly. 

\section{Version Control Log}

\begin{tabular}{lp{12cm}}
  \label{tabular:legend:git-log}
  \textbf{acronym} & \textbf{meaning} \\
  V & \texttt{version} \\
  tag & \texttt{git tag} \\
  MF & Number of \texttt{modified files}. \\
  AL & Number of \texttt{added lines}. \\
  DL & Number of \texttt{deleted lines}. \\
\end{tabular}

\bigskip

%\iflanguage{ngerman}{\shorthandoff{"}}{}
\begin{longtable}{|rllp{9cm}rrr|}
\hline \multicolumn{1}{|c}{\textbf{V}} & \multicolumn{1}{c}{\textbf{tag}}
& \multicolumn{1}{c}{\textbf{date}}
& \multicolumn{1}{c}{\textbf{commit message}} & \multicolumn{1}{c}{\textbf{MF}}
& \multicolumn{1}{c}{\textbf{AL}} & \multicolumn{1}{c|}{\textbf{DL}} \\ \hline
\endhead

\hline \multicolumn{7}{|r|}{\longtableendfoot} \\ \hline
\endfoot

\hline% \hline
\endlastfoot

\hline 1 &  & 2018-09-23 & Initial commit & 1 & 1 & 0 \\
\hline 2 &  & 2018-09-23 & add mersenne generator file, create concurrency file and makefile & 3 & 188 & 0 \\
\hline 3 &  & 2018-09-23 & add bufferNode struct & 0 & 0 & 0 \\
\hline 4 &  & 2018-09-25 & Minor addition, run command for makefile. & 1 & 1 & 0 \\
\hline 5 &  & 2018-09-25 & Fixed semicolon error with struct and modified makefile and include statements for mersenne.c & 2 & 15 & 10 \\
\hline 6 &  & 2018-09-25 & Added -pthread option on gcc command line & 1 & 1 & 1 \\
\hline 7 &  & 2018-09-25 & Added ability to create threads and produce and consume values, working on adding comments & 1 & 102 & 2 \\
\hline 8 &  & 2018-09-26 & Added initial comments to the code & 1 & 32 & 3 \\
\hline 9 &  & 2018-09-27 & Initial commit of LaTeX documentation files & 2 & 213 & 0 \\
\hline 10 &  & 2018-10-05 & Added inline assembly to support hardware randomization & 1 & 22 & 2 \\
\hline 11 &  & 2018-10-06 & Generate random number using rdrand32 & 2 & 5 & 3 \\
\hline 12 &  & 2018-10-11 & Updated the print statements for the number of things in the buffer. & 1 & 3 & 3 \\
\hline 13 &  & 2018-10-11 & Added .toc file to make clean & 1 & 1 & 1 \\
\hline 14 &  & 2018-10-11 & Updated formatting in document & 1 & 43 & 7 \\
\hline 15 &  & 2018-10-11 & Added abstract and remote target line for gdb & 1 & 3 & 2 \\
\hline 16 &  & 2018-10-11 & Added work log to documentation & 1 & 50 & 7 \\
\hline 17 &  & 2018-10-11 & Added clean option to makefile for concurrency & 1 & 4 & 0 \\
\hline 18 &  & 2018-10-11 & added latex-git-log files & 5 & 466 & 0 \\
\hline 19 &  & 2018-10-11 & Answered concurrency questions & 0 & 0 & 0 \\
\end{longtable}

\section{Work Log}

\textit{\textbf{23 September, 2018}} \\\\
\textbf{Kernel:} \\\\
\indent Today we attempted to get everything set up for the kernel. As the assignment had not been assigned yet we were trying to get a head start by looking through McGrath's assignment 1 description. We split up the work and all followed the command line options we wrote out at the top of this report. We attempted to start the emulator using the qemu command, we saw it stop and wait for execution from the GDB command. We were unable to connect to the tcp port with GDB at this point as the instructions seemed to be unclear to us.
\\\\
\textbf{Concurrency:} \\\\
\indent We started development on the concurrency assignment, working to get the Mersenne Twister to generate random values. We were able to do this, create a makefile, and get the git repository set up. We then started looking into and experimenting with different ways to output the git log into a table form.
\\\\

\noindent
\textit{\textbf{25 September, 2018}} \\\\
\textbf{Kernel:} \\
\indent Did not make any changes to kernel.
\\\\
\textbf{Concurrency:} \\
\indent Got the initial functionality working with a baseline makefile. These changes introduced the producer and consumer functionality to the script but some additional changes such as the rdrand32 needs to be implemented and code should be cleaned up. Previously we had added the struct and baseline c code to the file. Today we made changes that added the producer and consumer functionality. Comments need to be added.
\\\\

\noindent
\textit{\textbf{26 September, 2018}} \\\\
\textbf{Kernel:} \\
\indent No kernel changes.
\\\\
\textbf{Concurrency:} \\
\indent Added comments to concurrency code.
\\\\

\noindent
\textit{\textbf{27 September, 2018}} \\\\
\textbf{Documentation:} \\
\indent We then worked on our makefile for tex and committed both our makefile and an initial set of documentation written in tex.
\\\\

\noindent
\textit{\textbf{2 October, 2018}} \\\\
\textbf{Kernel:} \\
\indent We had struggled to run and connect GDB to the kernel that was waiting for us earlier. Today we asked questions during recitation and then went to office hours with the teacher to ask about it. Immediately we realized it was because prior to connecting with GDB we did not source the environment file. Once we found this out we were able to connect with GDB using \$GDB and had no problems.
\\\\
\textbf{Concurrency:} \\
\indent No concurrency changes.
\\\\

\noindent
\textit{\textbf{6/7 October, 2018}} \\\\
\textbf{Kernel:} \\
\indent No kernel changes.
\\\\
\textbf{Concurrency:} \\
\indent We investigated the rdrand32 and implemented it in the concurrency code.
\\\\

\noindent
\textit{\textbf{11 October, 2018}} \\\\
\textbf{Documentation:} \\
\indent We got together to work on completing the assignment to turn it in, we cleaned up the documentation, and added our work and commit log.
\\\\
\textbf{Concurrency:} \\
\indent Fixed a minor printing issue to make sure it prints out the buffer size properly.
\\\\

%}
\end{document}
